%
%	Default packages for memoir documents created by MultiMarkdown
%

\usepackage{fancyvrb}			% Allow \verbatim et al. in footnotes
\usepackage{graphicx}			% To enable including graphics in pdf's
\usepackage{booktabs}			% Better tables
\usepackage{tabulary}			% Support longer table cells
\usepackage[utf8]{inputenc}		% For UTF-8 support
\usepackage[T1]{fontenc}		% Use T1 font encoding for accented characters
\usepackage{xcolor}			% Allow for color (annotations)
\usepackage[sort&compress]{natbib} % Better bibliography support
\usepackage{fontspec}
\setmainfont[Mapping=tex-text]{VistaSlabReg}
\usepackage{color}
\usepackage{titlesec}
\usepackage{fancyhdr}
\pagestyle{fancy}

\voffset       -0.25in
\headheight     0.75in
\topmargin      -0.5in
\headsep        0.125in
\textheight     9.0in

\oddsidemargin  0in
\evensidemargin 0.0in
\textwidth      6.5in

\setlength\parindent{0.0in}

\definecolor{ocpblue}{rgb}{0.212,0.373,0.569}

\titleformat{\section}
{\color{ocpblue}\normalfont\Large\bfseries}{\thesection}{1em}{}

\fancyhead{}
\fancyhead[LO]{\mbox{\includegraphics[width=1.25in]
    {../images/OCPlogo_horiz.png}}}
%\fancyhead[RO]{Open Compute Project $\cdot$ Intel Motherboard $\cdot$ Hardware 2.0}
\fancyfoot{}
\makeatletter
\fancyfoot[LO]{http://opencompute.org}
\fancyfoot[RO]{\thepage}
\fancyfoot[LE]{\thepage}
\fancyfoot[RE]{\today}
\renewcommand\headrulewidth{0pt}
\renewcommand\footrulewidth{0pt}

\VerbatimFootnotes
